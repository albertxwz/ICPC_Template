\documentclass[10pt,UTF8,a4paper,twoside]{ctexart}

\usepackage{listings}
\usepackage{ctex}
\usepackage{multicol}
\usepackage{geometry}
\usepackage{graphicx}
\usepackage{subfigure}
\usepackage{fancyhdr}
\usepackage{appendix}
\usepackage{xcolor}
\usepackage{amsmath}
\usepackage{amssymb}

\geometry{a4paper,left=2cm,right=2cm,top=2cm,bottom=2cm}

% 用来设置附录中代码的样式
\lstset{
	language			=	C++,
    basicstyle          =   \ttfamily,          % 基本代码风格
    keywordstyle        =   \bfseries\color{blue},          % 关键字风格
    keywordstyle		=	[2] \color{teal},
    commentstyle        =   \rmfamily\itshape\color{red},  % 注释的风格,斜体
    stringstyle         =   \ttfamily\color{magenta},  % 字符串风格
    flexiblecolumns,                % 别问为什么,加上这个
    %numbers             =   left,   % 行号的位置在左边
    showspaces          =   false,  % 是否显示空格,显示了有点乱,所以不现实了
    %numberstyle         =   \zihao{-5}\ttfamily,    % 行号的样式,小五号,tt等宽字体
    showstringspaces    =   false,
    captionpos          =   t,      % 这段代码的名字所呈现的位置,t指的是top上面
    frame               =   single,   % 显示边框
    tabsize             =   4,
    breaklines          =   true,
    basewidth			=	0.5em,
}

\begin{document}

\begin{figure}[h]
\centering
\includegraphics{scut.jpg}
\end{figure}


%% temporary titles
% command to provide stretchy vertical space in proportion
\newcommand\nbvspace[1][3]{\vspace*{\stretch{#1}}}
% allow some slack to avoid under/overfull boxes
\newcommand\nbstretchyspace{\spaceskip0.5em plus 0.25em minus 0.25em}
% To improve spacing on titlepages
\newcommand{\nbtitlestretch}{\spaceskip0.6em}
\thispagestyle{empty}
\begin{center}
\bfseries
\nbvspace[1]
\Huge
\centering
{\nbtitlestretch\huge
ICPC TEMPLATE}

\begin{figure}[h]
\centering
\includegraphics[width=5cm,height=3cm]{ICPC.jpg}
\end{figure}

\nbvspace[1]
\normalsize


\nbvspace[1]
\small BY\\
\Large albertxwz\\[0.5em]
\footnotesize School of Computer Science \& Engineering,\\
South China University of Technology

\nbvspace[2]

%\includegraphics[width=1.5in]{./graphics/pic37}
\nbvspace[3]
\normalsize

Published in\\
\large
Dec 2021
\nbvspace[1]
\end{center}

\clearpage

\newpage
\begin{Large}
This template is a supplementary version.
\end{Large}
\mbox{}
\thispagestyle{empty}
\newpage
%*-----------------------------------------------------------------*
%|                             正文                       	     |
%*-----------------------------------------------------------------*
\begin{multicols}{2}
\columnseprule=0.5pt

\renewcommand \contentsname{Contents}
\tableofcontents
\thispagestyle{empty}

\clearpage


\setcounter{page}{1}
\pagestyle{headings}

\section{Standard Solution Template}
	\subsection{support bits/stdc++.h}
		\lstinputlisting[language=C++]{stdtemp/sol1.cpp}
	\subsection{unsupport bits/stdc++.h}
		\lstinputlisting[language=C++]{stdtemp/sol2.cpp}
	\subsection{Python Template}
		\lstinputlisting[language=python]{stdtemp/sol.py}
	
\clearpage

\section{Graph}
	\subsection{Network Flow}
		\subsubsection{Dinic}
			\lstinputlisting[language=C++]{graph/Dinic.cpp}
		\subsubsection{Edmonds-Karp}
			\lstinputlisting[language=C++]{graph/Edmonds-Karp.cpp}
	\subsection{DSU on Tree}
\clearpage

\section{Mathematics}
	\subsection{Number Theory}
		\subsubsection{Linear Inverse Modulo}
			\lstinputlisting[language=C++]{math/NumberTheory/LinInvMod.cpp}
		\subsubsection{Quick Power}
			\lstinputlisting[language=C++]{math/NumberTheory/powmod.cpp}
		\subsubsection{Baby-Step-Giant-Step Algorithm}
			$$ a^{x}\ \equiv \ b \ (mod \ n)  $$
			\lstinputlisting[language=C++]{math/NumberTheory/BSGS.cpp}
		\subsubsection{M$\ddot{o}$bius Inversion}
		\subsubsection{Dujiao Sieve}
			常见积性函数:
				$$\epsilon(n)=[n=1] $$
				$$I(n)=1 $$
				$$id(n)=n $$
				$$d(x)=\sum_{i|n}{1}$$
				$$\sigma(x)=\sum_{i|n}{i}$$
				$$\phi(i)=\sum_{i|n}{[gcd(x,i)=1]}$$
				\begin{equation}
					\nonumber
					\mu(x)=\left\{
					\begin{array}{rcl}
						1 & & {x=1}\\
						(-1)^k & & {\prod_{i=1}^{k}{q_i}=1}\\
						0 & & \max{\{q_i\}} > 1\\
					\end{array} \right.
				\end{equation} \\\\
			狄利克雷卷积:设$f, g$是两个数论函数,则有$(f\ast g)(n)=\sum_{d|n}{f(d)g(\frac{n}{d})}$ \\\\
			常见性质:
				$$ 1. \ \mu\ast I=\epsilon $$
				$$ 2. \ \phi\ast I=id $$
				$$ 3. \ \mu\ast id = \phi $$
			最后要有形如:
			$$g(1)S(1)=\sum_{i=1}^{n}{(f*g)(i)} - \sum_{i=2}^{n}{g(i)S(\lfloor \frac{n}{i} \rfloor)} $$
			\lstinputlisting[language=C++]{math/NumberTheory/Dujiao.cpp}
		\subsubsection{Chinese Remainder Theory}
			\lstinputlisting[language=C++]{math/NumberTheory/CRT.cpp}
		\subsubsection{Floor Sum}
			\lstinputlisting[language=C++]{math/NumberTheory/floorsum.cpp}
		
		\subsubsection{Min\_25 Sieve}
		\subsubsection{$\pi(x)$}
			$\pi(x)\sim \frac{x}{\ln x}$
			\lstinputlisting[language=C++]{math/NumberTheory/NumberofPrime.cpp}
		\subsubsection{Lucas' Theorem}
			\begin{lstlisting}[language=C++]
long long Lucas(long long n, long long m, long long p) {
	if (m == 0) return 1;
	return (C(n % p, m % p, p) * Lucas(n / p, m / p, p)) % p;
}
			\end{lstlisting}
	\subsection{Karatsuba Multiply}
		\lstinputlisting[language=C++]{math/karatsuba.cpp}
	\subsection{Fast Fourier Transform}
		\lstinputlisting[language=C++]{math/FFT.cpp}
	\subsection{Number Theory Transform}
		\lstinputlisting[language=C++]{math/NTT.cpp}
	\subsection{Lagrange Insertion Value Method}
		$$f(k)=\sum_{i=1}^{n}{y_i}\prod_{j \neq i}{\frac{k-x_j}{x_i-x_j}}$$
		\lstinputlisting[language=C++]{math/lagrange.cpp}
\clearpage

\section{Data Structure}
	\subsection{Treap}
		\lstinputlisting[language=C++]{DS/Treap.cpp}
	\subsection{Splay Tree}
	\subsection{Two-dimensional Segment Tree}
	\subsection{Mo's Algorithm}
		\lstinputlisting[language=C++]{DS/Mo'sAlgorithm.cpp}
\clearpage

\section{String}
	\subsection{KMP}
		\lstinputlisting[language=C++]{string/KMP.cpp}
	\subsection{Trie}
		\lstinputlisting[language=C++]{string/Trie.cpp}
\clearpage

\section{Computational Geometry}
	\subsection{2D Template}
		\lstinputlisting[language=C++]{CompGeo/TwoDim.cpp}
	\subsection{Smallest Covering Circle}
		\lstinputlisting[language=C++]{CompGeo/mincircle.cpp}
	\subsection{Half Plane Intersection}
		\lstinputlisting[language=C++]{CompGeo/HalfPlaneIntersect.cpp}
\clearpage

%*------------------------------------------------*
%|						附录						|
%*------------------------------------------------*

\begin{appendices}
	\section{Env Conf}
		\lstinputlisting{appendix/_vimrc}
		%\lstinputlisting{appendix/settings.json}
		\lstinputlisting{appendix/launch.json}
		\lstinputlisting{appendix/tasks.json}
		\lstinputlisting{appendix/c_cpp_properties.json}
		\lstinputlisting{appendix/CMakeLists.txt}
	
	\section{Theorem}
		\large
		\subsection{Lucas' Theorem}
			对于质数p,有
			$$
				\tbinom{n}{m} \ mod \ p = \tbinom{\lfloor n/p \rfloor}{\lfloor m/p \rfloor}
				\cdot \tbinom{n \ mod \ p}{m \ mod \ p} \ mod \ p
			$$
		\subsection{Betty's Theorem}
			如果两个无理数$a,b$满足:
			$$ \frac{1}{a}+\frac{1}{b}=1$$
			那么对于两个集合$A,B$:
			$$A=\lbrace \lbrack na \rbrack \rbrace,B=\lbrace \lbrack nb \rbrack \rbrace$$
			有下面两个结论:
			$$A \bigcap B = \emptyset,A \bigcup B = \mathbb{N}^+$$
		\subsection{Bernoulli Number}
			定义:$S(n, m) = \sum_{i=1}^{n}{i^m} \ (m > 0, \ n > 0)$
			则必有:$S(n, m)=\frac{1}{m+1}\sum_{i=0}^{m}{\tbinom{m+1}{i}B_{i}n^{m+1-i}}$
			其中:$B_n=-\frac{1}{n+1}\sum_{i=0}^{n-1}\tbinom{n+1}{i}B_i, \ B_0=1,\ B_1=\pm\frac{1}{2},\ B_2=\frac{1}{6},\ B_3=0,\ B_4=-\frac{1}{30}$ \\
			一般来说,自然数幂求和要$B_1$取正数。另外,伯努利数和$\zeta(s)$有关,
			$$ B_{2m} = \frac{2(-1)^{m-1}(2m)!}{(2\pi)^{2m}} \zeta(2m) $$
			使用多项式求逆或者分治fft可以快速计算伯努利数,因为伯努利数满足
			$$ \frac{x}{e^x-1} = \sum_{n=0}^{\infty}\frac{B_n}{n!}x^n = \frac{1}{\frac{e^x-1}{x}} $$
	\section{C++ STL: set}
		\large
		set与unordered\_set的区别在于有无在内部存储时有无顺序。
		\subsection{Basic Method}
			\input{appendix/stlset/basicmethod.txt}
		\subsection{Advanced Method}
			注:必须导入algorithm头文件
			\input{appendix/stlset/advance.txt}
		
\end{appendices}

\end{multicols}
\end{document}